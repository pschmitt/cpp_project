\section{C:/Users/siham/Desktop/Zylinder/Exemple\_\-Cylinder/Cylinder\-Library/src/cylinder.c-Dateireferenz}
\label{cylinder_8c}\index{C:/Users/siham/Desktop/Zylinder/Exemple_Cylinder/CylinderLibrary/src/cylinder.c@{C:/Users/siham/Desktop/Zylinder/Exemple\_\-Cylinder/CylinderLibrary/src/cylinder.c}}
{\tt \#include \char`\"{}cylinder.h\char`\"{}}\par
{\tt \#include \char`\"{}circle.h\char`\"{}}\par


Include-Abh\"{a}ngigkeitsdiagramm f\"{u}r cylinder.c:\subsection*{Funktionen}
\begin{CompactItemize}
\item 
double {\bf cylinder\-Capacity} (const {\bf Cylinder} $\ast$c)
\item 
double {\bf cylinder\-Lateral\-Area} (const {\bf Cylinder} $\ast$c)
\item 
double {\bf cylinder\-Surface} (const {\bf Cylinder} $\ast$c)
\end{CompactItemize}


\subsection{Dokumentation der Funktionen}
\index{cylinder.c@{cylinder.c}!cylinderCapacity@{cylinderCapacity}}
\index{cylinderCapacity@{cylinderCapacity}!cylinder.c@{cylinder.c}}
\subsubsection{\setlength{\rightskip}{0pt plus 5cm}double cylinder\-Capacity (const {\bf Cylinder} $\ast$ {\em c})}\label{cylinder_8c_c35ebde26ddb19866e3f1259e304cca8}


filename: {\bf cylinder.c}{\rm (S.\,\pageref{cylinder_8c})} 

Definiert in Zeile 5 der Datei cylinder.c.

Benutzt circle\-Area(), Cylinder::height und Cylinder::radius.

\footnotesize\begin{verbatim}6 {
7         if (!c || c->radius < 0 || c->height < 0) { return -1; }
8         return circleArea(c->radius) * c->height;
9 }
\end{verbatim}\normalsize 




Hier ist der Graph aller Aufrufe f\"{u}r diese Funktion:\index{cylinder.c@{cylinder.c}!cylinderLateralArea@{cylinderLateralArea}}
\index{cylinderLateralArea@{cylinderLateralArea}!cylinder.c@{cylinder.c}}
\subsubsection{\setlength{\rightskip}{0pt plus 5cm}double cylinder\-Lateral\-Area (const {\bf Cylinder} $\ast$ {\em c})}\label{cylinder_8c_bea49511e550ad0db24bf68ce91418ff}




Definiert in Zeile 19 der Datei cylinder.c.

Benutzt circumference(), Cylinder::height und Cylinder::radius.

\footnotesize\begin{verbatim}20 {
21         if (!c || c->radius < 0 || c->height < 0) { return -1; }
22         return circumference(c->radius) * c->height;
23 }
\end{verbatim}\normalsize 




Hier ist der Graph aller Aufrufe f\"{u}r diese Funktion:\index{cylinder.c@{cylinder.c}!cylinderSurface@{cylinderSurface}}
\index{cylinderSurface@{cylinderSurface}!cylinder.c@{cylinder.c}}
\subsubsection{\setlength{\rightskip}{0pt plus 5cm}double cylinder\-Surface (const {\bf Cylinder} $\ast$ {\em c})}\label{cylinder_8c_064d77300d0537b11e9d104340f2f959}




Definiert in Zeile 12 der Datei cylinder.c.

Benutzt circle\-Area(), cylinder\-Lateral\-Area(), Cylinder::height und Cylinder::radius.

\footnotesize\begin{verbatim}13 {
14         if (!c || c->radius < 0 || c->height < 0) { return -1; }
15         return 2 * circleArea(c->radius) + cylinderLateralArea(c);
16 }
\end{verbatim}\normalsize 




Hier ist der Graph aller Aufrufe f\"{u}r diese Funktion: