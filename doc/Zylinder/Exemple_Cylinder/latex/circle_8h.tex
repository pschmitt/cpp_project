\section{C:/Users/siham/Desktop/Zylinder/Exemple\_\-Cylinder/Cylinder\-Library/src/circle.h-Dateireferenz}
\label{circle_8h}\index{C:/Users/siham/Desktop/Zylinder/Exemple_Cylinder/CylinderLibrary/src/circle.h@{C:/Users/siham/Desktop/Zylinder/Exemple\_\-Cylinder/CylinderLibrary/src/circle.h}}


Dieser Graph zeigt, welche Datei direkt oder indirekt diese Datei enth\"{a}lt:\subsection*{Funktionen}
\begin{CompactItemize}
\item 
double {\bf circle\-Area} (double radius)
\item 
double {\bf circumference} (double radius)
\end{CompactItemize}
\subsection*{Variablen}
\begin{CompactItemize}
\item 
static const double {\bf PI} = 3.14
\end{CompactItemize}


\subsection{Dokumentation der Funktionen}
\index{circle.h@{circle.h}!circleArea@{circleArea}}
\index{circleArea@{circleArea}!circle.h@{circle.h}}
\subsubsection{\setlength{\rightskip}{0pt plus 5cm}double circle\-Area (double {\em radius})}\label{circle_8h_ef6c48edf38069b2b3684a790d806b92}


filename: {\bf circle.c}{\rm (S.\,\pageref{circle_8c})} 

Definiert in Zeile 4 der Datei circle.c.

Benutzt PI.

Wird benutzt von cylinder\-Capacity() und cylinder\-Surface().

\footnotesize\begin{verbatim}5 {
6         if (radius < 0) { return -1; }
7         return PI * radius * radius;
8 }
\end{verbatim}\normalsize 


\index{circle.h@{circle.h}!circumference@{circumference}}
\index{circumference@{circumference}!circle.h@{circle.h}}
\subsubsection{\setlength{\rightskip}{0pt plus 5cm}double circumference (double {\em radius})}\label{circle_8h_e48234f11885e6a380e29e50f272c7b4}




Definiert in Zeile 10 der Datei circle.c.

Benutzt PI.

Wird benutzt von cylinder\-Lateral\-Area().

\footnotesize\begin{verbatim}11 {
12         if (radius < 0) { return -1; }
13         return 2 * PI * radius;
14 }
\end{verbatim}\normalsize 




\subsection{Variablen-Dokumentation}
\index{circle.h@{circle.h}!PI@{PI}}
\index{PI@{PI}!circle.h@{circle.h}}
\subsubsection{\setlength{\rightskip}{0pt plus 5cm}const double {\bf PI} = 3.14\hspace{0.3cm}{\tt  [static]}}\label{circle_8h_3a372b04b95323b72c2b2f9a5d24e67b}


filename: {\bf circle.h}{\rm (S.\,\pageref{circle_8h})} 

Definiert in Zeile 5 der Datei circle.h.

Wird benutzt von circle\-Area() und circumference().