\section{C:/Users/siham/Desktop/Zylinder/Exemple\_\-Cylinder/Cylinder\-Application/src/cylinder\-IO.c-Dateireferenz}
\label{cylinder_i_o_8c}\index{C:/Users/siham/Desktop/Zylinder/Exemple_Cylinder/CylinderApplication/src/cylinderIO.c@{C:/Users/siham/Desktop/Zylinder/Exemple\_\-Cylinder/CylinderApplication/src/cylinderIO.c}}
{\tt \#include \char`\"{}cylinder\-IO.h\char`\"{}}\par
{\tt \#include $<$cylinder.h$>$}\par
{\tt \#include $<$stdio.h$>$}\par


Include-Abh\"{a}ngigkeitsdiagramm f\"{u}r cylinder\-IO.c:\subsection*{Funktionen}
\begin{CompactItemize}
\item 
int {\bf continuing} ()
\item 
int {\bf get\-Cylinder\-Data} ({\bf Cylinder} $\ast$c)
\item 
static double {\bf get\-Cylinder\-Height} ()
\item 
static double {\bf get\-Cylinder\-Radius} ()
\item 
void {\bf goodbye} ()
\item 
void {\bf on\-Wrong\-Input} ()
\item 
void {\bf print\-Cylinder\-Surface} (double surface)
\item 
void {\bf welcome} ()
\end{CompactItemize}


\subsection{Dokumentation der Funktionen}
\index{cylinderIO.c@{cylinder\-IO.c}!continuing@{continuing}}
\index{continuing@{continuing}!cylinderIO.c@{cylinder\-IO.c}}
\subsubsection{\setlength{\rightskip}{0pt plus 5cm}int continuing ()}\label{cylinder_i_o_8c_fbf29286f9288c2f6a6a903105442720}




Definiert in Zeile 27 der Datei cylinder\-IO.c.

Wird benutzt von main().

\footnotesize\begin{verbatim}28 {
29         int doit = 0;
30         char c = '\0'; 
31         printf("Weiter [J, j]: "); scanf("\n%c", &c);
32         if (c == 'j' || c == 'J') { doit = 1; }
33         return doit;
34 }
\end{verbatim}\normalsize 


\index{cylinderIO.c@{cylinder\-IO.c}!getCylinderData@{getCylinderData}}
\index{getCylinderData@{getCylinderData}!cylinderIO.c@{cylinder\-IO.c}}
\subsubsection{\setlength{\rightskip}{0pt plus 5cm}int get\-Cylinder\-Data ({\bf Cylinder} $\ast$ {\em c})}\label{cylinder_i_o_8c_b3241cde0b2a59c344dcec757f72a102}




Definiert in Zeile 41 der Datei cylinder\-IO.c.

Benutzt get\-Cylinder\-Height(), get\-Cylinder\-Radius(), Cylinder::height und Cylinder::radius.

Wird benutzt von main().

\footnotesize\begin{verbatim}42 {
43         int inputOK = 1;
44         if (!c) { return 0; }
45         c->radius = getCylinderRadius();
46         c->height = getCylinderHeight();
47         if (c->radius <= 0 || c->height <= 0) { inputOK = 0; }
48         return inputOK;
49 }
\end{verbatim}\normalsize 




Hier ist der Graph aller Aufrufe f\"{u}r diese Funktion:\index{cylinderIO.c@{cylinder\-IO.c}!getCylinderHeight@{getCylinderHeight}}
\index{getCylinderHeight@{getCylinderHeight}!cylinderIO.c@{cylinder\-IO.c}}
\subsubsection{\setlength{\rightskip}{0pt plus 5cm}static double get\-Cylinder\-Height ()\hspace{0.3cm}{\tt  [static]}}\label{cylinder_i_o_8c_585a90cce19544d7b64478ba46f8ad26}




Definiert in Zeile 14 der Datei cylinder\-IO.c.

Wird benutzt von get\-Cylinder\-Data().

\footnotesize\begin{verbatim}15 {
16         double height = -1.0; 
17         printf("Geben Sie bitte die Hoehe ein: "); 
18         scanf("%lf", &height);
19         return height;
20 }
\end{verbatim}\normalsize 


\index{cylinderIO.c@{cylinder\-IO.c}!getCylinderRadius@{getCylinderRadius}}
\index{getCylinderRadius@{getCylinderRadius}!cylinderIO.c@{cylinder\-IO.c}}
\subsubsection{\setlength{\rightskip}{0pt plus 5cm}static double get\-Cylinder\-Radius ()\hspace{0.3cm}{\tt  [static]}}\label{cylinder_i_o_8c_9fc989da911ad0ede3644d16b086f4fc}


filename: {\bf cylinder\-IO.c}{\rm (S.\,\pageref{cylinder_i_o_8c})} 

Definiert in Zeile 6 der Datei cylinder\-IO.c.

Wird benutzt von get\-Cylinder\-Data().

\footnotesize\begin{verbatim}7 {
8         double radius = -1.0; 
9         printf("Geben Sie bitte den Radius ein: "); 
10         scanf("%lf", &radius);
11         return radius;
12 }
\end{verbatim}\normalsize 


\index{cylinderIO.c@{cylinder\-IO.c}!goodbye@{goodbye}}
\index{goodbye@{goodbye}!cylinderIO.c@{cylinder\-IO.c}}
\subsubsection{\setlength{\rightskip}{0pt plus 5cm}void goodbye ()}\label{cylinder_i_o_8c_9d02e3c6092fbd61a94d4bda90de3fc5}




Definiert in Zeile 36 der Datei cylinder\-IO.c.

Wird benutzt von main().

\footnotesize\begin{verbatim}37 {
38         printf("\nCopyright: HTW des Saarlandes\n");
39 }
\end{verbatim}\normalsize 


\index{cylinderIO.c@{cylinder\-IO.c}!onWrongInput@{onWrongInput}}
\index{onWrongInput@{onWrongInput}!cylinderIO.c@{cylinder\-IO.c}}
\subsubsection{\setlength{\rightskip}{0pt plus 5cm}void on\-Wrong\-Input ()}\label{cylinder_i_o_8c_ac5129efd394503c2590d23da2c37561}




Definiert in Zeile 57 der Datei cylinder\-IO.c.

Wird benutzt von main().

\footnotesize\begin{verbatim}58 {
59         printf("FEHLER: Bitte Beachten: Radius und Hoehe positiv! \n");
60 }
\end{verbatim}\normalsize 


\index{cylinderIO.c@{cylinder\-IO.c}!printCylinderSurface@{printCylinderSurface}}
\index{printCylinderSurface@{printCylinderSurface}!cylinderIO.c@{cylinder\-IO.c}}
\subsubsection{\setlength{\rightskip}{0pt plus 5cm}void print\-Cylinder\-Surface (double {\em surface})}\label{cylinder_i_o_8c_d946199312576e1ff9d3ac37a8bedacf}




Definiert in Zeile 52 der Datei cylinder\-IO.c.

Wird benutzt von main().

\footnotesize\begin{verbatim}53 {
54         printf("Die Oberflaeche des Zylinders betraegt %f\n", surface);
55 }
\end{verbatim}\normalsize 


\index{cylinderIO.c@{cylinder\-IO.c}!welcome@{welcome}}
\index{welcome@{welcome}!cylinderIO.c@{cylinder\-IO.c}}
\subsubsection{\setlength{\rightskip}{0pt plus 5cm}void welcome ()}\label{cylinder_i_o_8c_fdf71b0239b7d4a9b67b784177146385}




Definiert in Zeile 22 der Datei cylinder\-IO.c.

Wird benutzt von main().

\footnotesize\begin{verbatim}23 {
24         printf("Oberflaeche eines Zylinders!\n\n");
25 }
\end{verbatim}\normalsize 


